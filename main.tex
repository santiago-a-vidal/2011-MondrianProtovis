% % % % % % % % % % % % % % % % % % % % % % % % % % % % % % % % % %
%\documentclass[runningheads]{llncs}
\documentclass[preprint,10pt]{sigplanconf}

% packages
\usepackage{xspace}
\usepackage{ifthen}
\usepackage{amsbsy}
\usepackage{amssymb}
\usepackage{balance}
\usepackage{booktabs}
\usepackage{graphicx}
\usepackage{multirow}
\usepackage{needspace}
\usepackage{microtype}
\usepackage{bold-extra}
\usepackage{array}
\usepackage{epstopdf}


% references
\usepackage[colorlinks]{hyperref}
\usepackage[all]{hypcap}
\setcounter{tocdepth}{2}
\hypersetup{
	colorlinks=true,
	urlcolor=black,
	linkcolor=black,
	citecolor=black,
	plainpages=false,
	bookmarksopen=true}

\def\chapterautorefname{Chapter}
\def\appendixautorefname{Appendix}
\def\sectionautorefname{Section}
\def\subsectionautorefname{Section}
\def\figureautorefname{Figure}
\def\tableautorefname{Table}
\def\listingautorefname{Listing}

% source code
\usepackage{xcolor}
\usepackage{textcomp}
\usepackage{listings}
\definecolor{source}{gray}{0.9}
\lstset{
	language={},
	% characters
	tabsize=3,
	upquote=true,
	escapechar={!},
	keepspaces=true,
	breaklines=true,
	alsoletter={\#:},
	breakautoindent=true,
	columns=fullflexible,
	showstringspaces=false,
	basicstyle=\footnotesize\ttfamily,
	% background
	frame=single,
    framerule=0pt,
	backgroundcolor=\color{source},
	% numbering
	numbersep=5pt,
	numberstyle=\tiny,
	numberfirstline=true,
	% captioning
	captionpos=b,
	% formatting (html)
	moredelim=[is][\textbf]{<b>}{</b>},
	moredelim=[is][\textit]{<i>}{</i>},
	moredelim=[is][\color{red}\uwave]{<u>}{</u>},
	moredelim=[is][\color{red}\sout]{<del>}{</del>},
	moredelim=[is][\color{blue}\underline]{<ins>}{</ins>}}
\newcommand{\ct}{\lstinline[backgroundcolor=\color{white},basicstyle=\footnotesize\ttfamily]}
\newcommand{\lct}[1]{{\small\tt #1}}

% tikz
% \usepackage{tikz}
% \usetikzlibrary{matrix}
% \usetikzlibrary{arrows}
% \usetikzlibrary{external}
% \usetikzlibrary{positioning}
% \usetikzlibrary{shapes.multipart}
% 
% \tikzset{
% 	every picture/.style={semithick},
% 	every text node part/.style={align=center}}
% \tikzexternalize[prefix=figures/]{quality}

% proof-reading
\usepackage{xcolor}
\usepackage[normalem]{ulem}
\newcommand{\ra}{$\rightarrow$}
\newcommand{\ugh}[1]{\textcolor{red}{\uwave{#1}}} % please rephrase
\newcommand{\ins}[1]{\textcolor{blue}{\uline{#1}}} % please insert
\newcommand{\del}[1]{\textcolor{red}{\sout{#1}}} % please delete
\newcommand{\chg}[2]{\textcolor{red}{\sout{#1}}{\ra}\textcolor{blue}{\uline{#2}}} % please change
\newcommand{\chk}[1]{\textcolor{ForestGreen}{#1}} % changed, please check

% comments \nb{label}{color}{text}
\newboolean{showcomments}
\setboolean{showcomments}{true}
\ifthenelse{\boolean{showcomments}}
	{\newcommand{\nb}[3]{
		{\colorbox{#2}{\bfseries\sffamily\scriptsize\textcolor{white}{#1}}}
		{\textcolor{#2}{\sf\small$\blacktriangleright$\textit{#3}$\blacktriangleleft$}}}
	 \newcommand{\version}{\emph{\scriptsize$-$Id$-$}}}
	{\newcommand{\nb}[2]{}
	 \newcommand{\version}{}}
\newcommand{\rev}[2]{\nb{Reviewer #1}{red}{#2}}
\newcommand{\ab}[1]{\nb{Alexandre}{blue}{#1}}
\newcommand{\sv}[1]{\nb{Santiago}{orange}{#1}}

% graphics: \fig{position}{percentage-width}{filename}{caption}
\DeclareGraphicsExtensions{.png,.jpg,.pdf,.eps,.gif}
%\graphicspath{{figures/}}
\newcommand{\fig}[4]{
	\begin{figure}[#1]
		\centering
		\includegraphics[width=#2\textwidth]{#3}
		\caption{\label{fig:#3}#4}
	\end{figure}}
\newcommand{\largefig}[4]{
	\begin{figure*}[#1]
		\centering
		\includegraphics[width=#2\textwidth]{#3}
		\caption{\label{fig:#3}#4}
	\end{figure*}}

% abbreviations
\newcommand{\ie}{\emph{i.e.,}\xspace}
\newcommand{\eg}{\emph{e.g.,}\xspace}
\newcommand{\etc}{\emph{etc.}\xspace}
\newcommand{\etal}{\emph{et al.}\xspace}

% lists
\newenvironment{bullets}[0]
	{\begin{itemize}}
	{\end{itemize}}

\newcommand{\seclabel}[1]{\label{sec:#1}}
\newcommand{\secref}[1]{Section~\ref{sec:#1}\xspace}
\newcommand{\figlabel}[1]{\label{fig:#1}}
\newcommand{\figref}[1]{Figure~\ref{fig:#1}\xspace}

% D O C U M E N T
% % % % % % % % % % % % % % % % % % % % % % % % % % % % % % % % % %
\begin{document}

% T I T L E
% % % % % % % % % % % % % % % % % % % % % % % % % % % % % % % % % %

\title{Extending Mondrian with Interactive HTML Visualisation}

\authorinfo{Santiago Vidal}
	{ISISTAN Research Institute, Faculty of Sciences, UNICEN University, Campus Universitario, Tandil, Buenos Aires, Argentina, Also CONICET}
	{svidal@exa.unicen.edu.ar}
\authorinfo{Alexandre Bergel}
	{PLEIAD Lab, Department of Computer Science (DCC), University of Chile, Santiago, Chile}
	{http://bergel.eu}
\authorinfo{Claudia Marcos} 
	{ISISTAN Research Institute, Faculty of Sciences, UNICEN University, Campus Universitario, Tandil, Buenos Aires, Argentina, Also CIC}
	{cmarcos@exa.unicen.edu.ar}


\maketitle

% A B S T R A C T
% % % % % % % % % % % % % % % % % % % % % % % % % % % % % % % % % %

\begin{abstract}
\end{abstract}

%: % % % % % % % % % % % % % % % % % % % % % % % % % % % % % % % % %
\section{Introduction}\seclabel{introduction}

%problem

%solution

%surprising result

%contribution

%\item lesson learnt

%outline
The paper is structured as follows.
%\secref{problem} shows the problem we faced when trying to evolve Mondrian.
%\secref{refactoring} describes the aspect-based solution we adopted.
%\secref{results} presents the impact of our solution on Mondrian.
%\secref{relatedWork} briefly revise the related work.
%\secref{conclusion} concludes.


%: % % % % % % % % % % % % % % % % % % % % % % % % % % % % % % % % %
%\section{Refactoring Mondrian to Enable the Extensions}\seclabel{problem}
%
%Implementation of the visitor pattern instead of the displayOn: methods
%Show some benchmarks. We have they in a mail with the subject bench

\section{Protovis}
Short introduction of Protovis

%: % % % % % % % % % % % % % % % % % % % % % % % % % % % % % % % % %
\section{Generating HTML Files using a Visualization Library}\seclabel{refactoring}

\ab{We need some example. For example, how do you translate ``view nodes: MOShape withAllSubclasses''}
\ab{view shape rectangle height: \#numberOfMethods; width: \#numberOfAttributes. view nodes: MOShape withAllSubclasses}

\ab{Layout? Popup?}

Explain the implementation of the Visitor pattern that collects the data to generate the HTML file.
Also, what are the main benefics of using this kind of visualization.
Supported graphs??
Layouts
%I used the layouts given by Mondrian
labels and popup. 
%I used Tipsy
Handling interaction.
%I reimplemented drag&drop
solving performance issues
% Some things changed in the drag & drop implementation
Packaging the library files
%: % % % % % % % % % % % % % % % % % % % % % % % % % % % % % % % % %
\section{Future Works}\seclabel{results}
being behind a seaside server


%: % % % % % % % % % % % % % % % % % % % % % % % % % % % % % % % % %
\section{Conclusion}\seclabel{conclusion}



% % % % % % % % % % % % % % % % % % % % % % % % % % % % % % % % % %
%\section*{Acknowledgments}
%
%\small We gratefully thanks ...

% bibliography
% % % % % % % % % % % % % % % % % % % % % % % % % % % % % % % % %
\bibliographystyle{abbrvnat}
\bibliography{}

\end{document}
